\documentclass[journal,12pt,twocolumn]{IEEEtran}

\usepackage{setspace}
\usepackage{gensymb}
\singlespacing
\usepackage[cmex10]{amsmath}

\usepackage{amsthm}
\usepackage{amsmath}
\usepackage{amssymb}
\usepackage{mathrsfs}
\usepackage{txfonts}
\usepackage{stfloats}
\usepackage{bm}
\usepackage{cite}
\usepackage{cases}
\usepackage{subfig}

\usepackage{longtable}
\usepackage{multirow}

\usepackage{enumitem}
\usepackage{mathtools}
\usepackage{steinmetz}
\usepackage{tikz}
\usepackage{circuitikz}
\usepackage{verbatim}
\usepackage{tfrupee}
\usepackage[breaklinks=true]{hyperref}
\usepackage{graphicx}
\usepackage{tkz-euclide}

\usetikzlibrary{calc,math}
\usepackage{listings}
    \usepackage{color}                                            %%
    \usepackage{array}                                            %%
    \usepackage{longtable}                                        %%
    \usepackage{calc}                                             %%
    \usepackage{multirow}                                         %%
    \usepackage{hhline}                                           %%
    \usepackage{ifthen}                                           %%
    \usepackage{lscape}     
\usepackage{multicol}
\usepackage{chngcntr}

\DeclareMathOperator*{\Res}{Res}

\renewcommand\thesection{\arabic{section}}
\renewcommand\thesubsection{\thesection.\arabic{subsection}}
\renewcommand\thesubsubsection{\thesubsection.\arabic{subsubsection}}

\renewcommand\thesectiondis{\arabic{section}}
\renewcommand\thesubsectiondis{\thesectiondis.\arabic{subsection}}
\renewcommand\thesubsubsectiondis{\thesubsectiondis.\arabic{subsubsection}}


\hyphenation{op-tical net-works semi-conduc-tor}
\def\inputGnumericTable{}                                 %%

\lstset{
%language=C,
frame=single, 
breaklines=true,
columns=fullflexible
}
\begin{document}


\newtheorem{theorem}{Theorem}[section]
\newtheorem{problem}{Problem}
\newtheorem{proposition}{Proposition}[section]
\newtheorem{lemma}{Lemma}[section]
\newtheorem{corollary}[theorem]{Corollary}
\newtheorem{example}{Example}[section]
\newtheorem{definition}[problem]{Definition}

\newcommand{\BEQA}{\begin{eqnarray}}
\newcommand{\EEQA}{\end{eqnarray}}
\newcommand{\define}{\stackrel{\triangle}{=}}
\bibliographystyle{IEEEtran}
\raggedbottom
\setlength{\parindent}{0pt}
\providecommand{\mbf}{\mathbf}
\providecommand{\pr}[1]{\ensuremath{\Pr\left(#1\right)}}
\providecommand{\qfunc}[1]{\ensuremath{Q\left(#1\right)}}
\providecommand{\sbrak}[1]{\ensuremath{{}\left[#1\right]}}
\providecommand{\lsbrak}[1]{\ensuremath{{}\left[#1\right.}}
\providecommand{\rsbrak}[1]{\ensuremath{{}\left.#1\right]}}
\providecommand{\brak}[1]{\ensuremath{\left(#1\right)}}
\providecommand{\lbrak}[1]{\ensuremath{\left(#1\right.}}
\providecommand{\rbrak}[1]{\ensuremath{\left.#1\right)}}
\providecommand{\cbrak}[1]{\ensuremath{\left\{#1\right\}}}
\providecommand{\lcbrak}[1]{\ensuremath{\left\{#1\right.}}
\providecommand{\rcbrak}[1]{\ensuremath{\left.#1\right\}}}
\theoremstyle{remark}
\newtheorem{rem}{Remark}
\newcommand{\sgn}{\mathop{\mathrm{sgn}}}
\providecommand{\abs}[1]{\left\vert#1\right\vert}
\providecommand{\res}[1]{\Res\displaylimits_{#1}} 
\providecommand{\norm}[1]{\left\lVert#1\right\rVert}
%\providecommand{\norm}[1]{\lVert#1\rVert}
\providecommand{\mtx}[1]{\mathbf{#1}}
\providecommand{\mean}[1]{E\left[ #1 \right]}
\providecommand{\fourier}{\overset{\mathcal{F}}{ \rightleftharpoons}}
%\providecommand{\hilbert}{\overset{\mathcal{H}}{ \rightleftharpoons}}
\providecommand{\system}{\overset{\mathcal{H}}{ \longleftrightarrow}}
	%\newcommand{\solution}[2]{\textbf{Solution:}{#1}}
\newcommand{\solution}{\noindent \textbf{Solution: }}
\newcommand{\cosec}{\,\text{cosec}\,}
\providecommand{\dec}[2]{\ensuremath{\overset{#1}{\underset{#2}{\gtrless}}}}
\newcommand{\myvec}[1]{\ensuremath{\begin{pmatrix}#1\end{pmatrix}}}
\newcommand{\mydet}[1]{\ensuremath{\begin{vmatrix}#1\end{vmatrix}}}
\numberwithin{equation}{subsection}
\makeatletter
\@addtoreset{figure}{problem}
\makeatother
\let\StandardTheFigure\thefigure
\let\vec\mathbf
\renewcommand{\thefigure}{\theproblem}
\def\putbox#1#2#3{\makebox[0in][l]{\makebox[#1][l]{}\raisebox{\baselineskip}[0in][0in]{\raisebox{#2}[0in][0in]{#3}}}}
     \def\rightbox#1{\makebox[0in][r]{#1}}
     \def\centbox#1{\makebox[0in]{#1}}
     \def\topbox#1{\raisebox{-\baselineskip}[0in][0in]{#1}}
     \def\midbox#1{\raisebox{-0.5\baselineskip}[0in][0in]{#1}}
\vspace{3cm}
\title{Assignment 1}
\author{Gaureesha Kajampady - EP20BTECH11005}
\maketitle
\newpage
\bigskip
\renewcommand{\thefigure}{\theenumi}
\renewcommand{\thetable}{\theenumi}
Download all python codes from 
\begin{lstlisting}
https://github.com/gaureeshk/AI1103/blob/main/Codes/assignment1.py
\end{lstlisting}
%
and latex-tikz codes from 
%
\begin{lstlisting}
https://github.com/gaureeshk/AI1103/blob/main/assignment1.tex
\end{lstlisting}
\section{Problem}
 A card from a pack of 52 cards is lost. From
the remaining cards of the pack, two cards
are drawn and are found to be both diamonds.
Find the probability of the lost card being a
diamond.

\section{Solution}
Let \textbf{A} be the event of a diamond card becoming lost\\
Then \textbf{A'}, which is the complement of A will be the event of a card which is not diamond becoming lost.\\
Let \textbf{B} be the event of getting 2 diamonds in the 2 draws.\\
The required probability is pr(A$|$B).\\
\\Since there are 13 diamond cards,
\begin{align}
    pr(A)=\frac{13}{52}=\frac{1}{4}\\
    \implies pr(A')=1-pr(A)=\frac{3}{4}
\end{align}
AB is the event of a diamond card getting lost and getting 2 diamond cards in the 2 draws.\\
Hence,
\begin{align}
    pr(AB)=\frac{\binom{13}{3}}{\binom{52}{3}}=\frac{13!\:49!}{10!\:52!}
\end{align}
We also know that,
\begin{align}
    pr(B)=pr(B \vert A)pr(A)+pr(B \vert A')pr(A') \label{eq1}
\end{align}
 pr(B$|$A) is probability of selecting 2 diamond cards given that one diamond card is lost.
\begin{align}
   \implies pr(B \vert A)=\frac{\binom{12}{2}}{\binom{51}{2}}=\frac{12!\:49!}{10!\:51!}
\end{align}
pr(B$|$A') is probability of selecting 2 diamond cards given that the card lost is not a diamond.
\begin{align}
    \implies pr(B \vert A')=\frac{\binom{13}{2}}{\binom{51}{2}}=\frac{13!\:49!}{11!\:51!}
\end{align}
by using equation \ref{eq1},
\begin{align}
\begin{split}
    pr(B)&=\frac{12!\:49!}{10!\:51!\:4}+\frac{13!\:49!\:3}{11!\:51!\:4}\\\\
         &=\frac{12!\:49!\:11}{11!\:51!\:4}+\frac{12!\:49!\:39}{11!\:51!\:4}\\\\
         &=\frac{12!\:49!\:50}{11!\:51!\:4}
\end{split}
\end{align}
by definition,
\begin{align}
\begin{split}
    pr(A \vert B)&=\frac{pr(AB)}{pr(B)}\\\\
                 &=\frac{13!\:49!\:11!\:51!\:4}{10!\:52!\:12!\:49!\:50}\\\\
                 &=\frac{13*11*4}{52*50}\\\\
                 &=\frac{11}{50}\\\\
                 &=0.22
\end{split}
\end{align}
Hence the probability of the lost card being a diamond (given that the 2 cards drawn are diamonds) is \textbf{0.22}.
\end{document}





















